
% Thesis Abstract -----------------------------------------------------


%\begin{abstractslong}    %uncommenting this line, gives a different abstract heading
\begin{abstracts}        %this creates the heading for the abstract page
Nonlinear cointegration model has been a popular tool for applied econometric modelling. There are numerous real life time series examples that demonstrate nonlinear response to another nonstationary time series in the field of macro-economics. The nonparametric estimation methods for the nonlinear linked function have been studied extensively in the literature. Most of the existing studies concentrate on establishing point-wise convergence, and there is little research on uniform convergence. The uniform convergence of nonparametric estimator of the nonlinear function is an important theoretic tool for further extension to the nonlinear cointegration model, including incorporating a time varying nonstationary error structure.

The aim of this work is to establish the uniform convergence of the nonparametric estimator of cointegrating function. Our results provide sharp convergence rate and  optimal range in which the uniform convergence holds. We systematically investigate the convergence of a class of martingale and a functional of nonstationary time series, which are key tools for the investigation of the nonparametric estimators.  Our framework supports for a wide choice of popular time series, including stationary mixing time series, stationary iterated random functions, Harris recurrent Markov chains and $I$(1) process with innovations being general linear processes. Additionally, the established asymptotics has an application in unit root testing with nonstationary volatility.

Furthermore, we also discuss the nonlinear parametric cointegration model. We establish a general framework for weak consistency of the nonlinear least squares (NLS) estimator that is easy to apply for various nonstationary time series, including partial sum of linear processes and Harris recurrent Markov chains. We also provide limit distributions for the NLS estimator, extending the previous works. Finally, we introduce endogeneity to the parametric model, by allowing the error sequence to be serially dependent and cross correlated with the regressors.


\end{abstracts}
%\end{abstractlongs}


% ----------------------------------------------------------------------


%%% Local Variables: 
%%% mode: latex
%%% TeX-master: "../thesis"
%%% End: 
