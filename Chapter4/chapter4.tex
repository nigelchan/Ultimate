\chapter{Application to Nonlinear Cointegration Model}
\ifpdf
    \graphicspath{{Chapter3/Chapter3Figs/PNG/}{Chapter3/Chapter3Figs/PDF/}{Chapter3/Chapter3Figs/}}
\else
    \graphicspath{{Chapter3/Chapter3Figs/EPS/}{Chapter3/Chapter3Figs/}}
\fi

\section{Introduction}

Consider a non-linear cointegrating regression model:
\begin{equation}
y_{t}=m(x_{t})+u_{t},\quad t=1,2,...,n,  \label{mo1}
\end{equation}%
where $u_{t}$ is a stationary error process and $x_{t}$ is a
non-stationary
regressor. Let $K(x)$ be a non-negative real function and set $%
K_{h}(s)=h^{-1}K(s/h)$ where $h\equiv h_{n}\rightarrow 0$. The
conventional kernel estimate of $m(x)$ in model (\ref{mo1}) is given
by
\begin{equation}
\hat{m}(x)=\frac{\sum_{t=1}^{n}y_{t}K_{h}(x_{t}-x)}{%
\sum_{t=1}^{n}K_{h}(x_{t}-x)}.  \label{add9}
\end{equation}%
The point-wise  limit behavior of $\hat{m}(x)$ has currently been investigated
by many authors. Among them, Karlsen, et al. (2007) discussed the situation where $x_{t}$ is a recurrent Markov chain.  Wang and Phillips (2009a, 2011) and Cai, et al.
(2009) considered an alternative treatment by making use of local
time limit theory and, instead of recurrent Markov chains, worked
with partial
 sum representations of the type
 $x_t=\sum_{j=1}^t\xi_j$ where $\xi_j$ is a general linear process.
 In  another paper, Wang and Phillips (2009b) considered
  the errors $u_t$  to be serially dependent and
 cross correlated with the regressor $x_t$ for small lags.  Also, see Phillips and Park (1998), Karlsen and Tj\o stheim (2001), Guerre (2004) and Bandi (2004) for
related work on non-linear, non-stationary autoregressions.

This section provides a uniform convergence for the  $\hat{m}(x)$ by making direct use of Theorems 2.1 and 2.3 in developing the asymptotics. 

\section{Main Results}
For reading convenience, we list the assumptions as follows.
\vskip 0.3cm
 \textbf{Assumption 3.1.} (i) $\{x_k\}_{k\ge 0}$ is a $\beta$-regular Harris recurrent Markov chain defined as in Section 3,
where   the invariant measure $\pi$ has a bounded  density function $p(s)$ on $R$;
(ii) $\{u_t, {\mathcal F}_t\}_{t\ge 1}$ is a martingale difference, where ${\mathcal F}_t=\si (x_1, ..., x_{t+1}, u_1,...,u_t)$, satisfying
 $E(u_t^2\mid {\mathcal F}_{t-1})\to_{a.s.} \si^2<\infty$ and $ \sup_{t\ge 1}E(|u_t|^{2p}\mid {\mathcal F}_{t-1})<\infty$, where  $p\ge 1+1/\ep_0$ for some $0<\ep_0<\beta$.



 \vskip 0.3cm \textbf{Assumption 3.2.} The kernel $K$ satisfies that $\int_{-\infty}^{\infty}K(s)ds<\infty$, $\sup_xK(x)<\infty$
 and for any $x, y \in R$,
$$
|K(x)-K(y)| \le C\, |x-y|.
$$

\vskip 0.3cm \textbf{Assumption 3.3.}
There   exists a real positive function $g(x)$ such that
\bestar
|m(y)-m(x)| &\leq& C\,|y-x|^{\alpha} g(x),\eestar
uniformly for some $0<\al\le 1$ and  any  $(x, y)\in \Omega_\ep$, where $\ep$ can be chosen sufficient small and
$
\Omega_{\ep} = \{(x,y): |y-x|\le \ep, x\in R\}.
$


\vskip 0.3cm
Assumption 3.1 is similar to, but weaker than  those appeared in Karlsen, et al. (2007), where  the authors considered the point-wise convergence in distribution.

Assumption 3.2 is a standard condition on $K(x)$  as in the stationary
situation. The Lipschitz condition on $K(x)$ is not necessary if we
only investigate the point-wise asymptotics. See Remark 3.2 for
further details.

Assumption 3.3 requires a Lipschitz-type condition in a small
neighborhood of the targeted  set for the functionals to be
estimated. This condition is quite weak, which may host a wide set
of functionals. Typical examples  include that $m(x)=\theta_1+\theta_2x+...+\theta_kx^{k-1}$;
 $m(x)=\al+ \beta\, x^{\gamma}$;
 $m(x)=x(1+\theta x)^{-1}I(x\ge 0)$;
 $m(x)=(\al+\beta\, e^{x})/(1+e^x)$.



\medskip

We have the following asymptotic results.

\begin{thm} \la {th31} Suppose Assumptions 3.1-3.3 hold, $h\to 0$ and $n^{-\ep_0}a(n)h\to \infty$ where $0<\ep_0<\beta$  is given as in Assumption 3.1.
It follows that
\begin{equation}
\sup_{|x|\le b_n'}|\hat{m}(x)-m(x)|=
O_{P}\left\{\big[a(n)h\big]^{-1/2}\,\log^{1/2}n
+h^{\alpha}\, \delta_n\right\},
\label{q1}\end{equation}
where $b_n'\le b_n$, $\delta_n=\sup_{|x|\le b_n'}g(x)$ and $b_n$ satisfies that
\bestar
\inf_{|x|\le b_n+1}  \sum_{k=1}^n E K[(x_k+x)/h]\ge a(n)\, h/C_0,
\eestar
for some $C_0>0$ and all $n$ sufficiently large. In particular, for the random walk $x_t$  defined as in Corollary \ref {th3}, we have
\begin{equation}
\sup_{|x|\le b_n'}|\hat{m}(x)-m(x)|=
O_{P}\left\{\big(nh^{2}\big)^{-1/4}\,\log^{1/2}n
+h^{\alpha}\, \delta_n\right\},
\label{q1a}\end{equation}
where $b_n'\le M_0\sqrt n$ for a fixed $M_0>0$ and $\delta_n=\sup_{|x|\le b_n'}g(x)$.
\end{thm}

\medskip
\begin{rem} When a high moment exists on the error $u_t$,
the $\ep_0$ can be chosen sufficient small so that there are more bandwidth choices  in practice. It is understandable that the results (\ref {q1})
 and (\ref {q1a}) are meaningful if only $h^{\al}\delta_n\to 0$, which depends on the tail of the unknown regression function $m(x)$, the bandwidth $h$ and the range $|x|\le b_n'$.
When $m(x)$ has a light tail such as $m(x)=(\al+\beta\, e^{x})/(1+e^x)$, $\delta_n$ may be bounded by a constant. In this situation, the $b_n'$ in (\ref {q1a}) can be chosen to be $M_0\sqrt n$ for some fixed $M_0>0$.
In contrast to Theorem 2.3 and Remark 2.3, this kind of range $|x|\le M_0\sqrt n$ might be optimal, that is, the $b_n'$ cannot be extended to $b_n'/\sqrt n \to\infty$ to establish the same rate of convergence as in (\ref {q1a}).

\end{rem}

\medskip
\begin{rem} Both results (\ref {q1}) and (\ref {q1a}) are sharp. However, a better result can be obtained if we are only interested in the point-wise asymptotics for $\hat m(x)$. For instance, as in Wang and Phillips (2009a,
b) with minor modification, we may show that, for each  $x$, \be \hat m(x)-m(x) &=& O_{P}
\left\{(nh^2)^{-1/4}+h^{\alpha}\right\},\ee
whenever  $x_t$ is a random walk defined as in Corollary \ref {th3}.
Furthermore $\hat m(x)$ has an asymptotic distribution that is
mixing normal, under minor additional conditions. More details are  referred to Wang and Phillips (2009a, b).

\end{rem}

\medskip
\begin{rem} Wang and Wang (2011) established a similar result to (\ref {q1a}) with the $x_t$ being a partial sum of linear process, but only for the $x$ being a compact support and imposing  a bounded condition on $u_t$. The setting on the $x_t$ in this paper is similar to that given in Gao, et al. (2011), but our result provides the optimal range
 for the uniform convergence holding true and removes  the independence between the error $u_t$ and $x_t$ required by Gao, et al. (2011).

\end{rem}


%%% Local Variables: 
%%% mode: latex
%%% TeX-master: "../thesis"
%%% End: 
