\chapter{Local Time Process} \la{chap:app2}

This appendix gives the definition and some basic results on the local time of a continuous Gaussian process. The process $\{L_G(t,x), t\ge 0, x \in R\}$ is called the local time of a continuous Gaussian process $\{G(t), t\ge 0\}$ if, for any locally integrable function $T$,
\bestar
\int_{0}^{t} T[G(s)] ds = \int_{-\infty}^{\infty} T(s) L_G(t, s) ds, \quad \mbox{all } t \in  R,
\eestar
with probability one. Such quantity can be interpreted as the time spent by the process $G$ at the spatial point $x$ over the time interval $[0, t]$.

The following theorem from Theorem 22.1 of \cite{gemanhorowitz1980} guarantees the existence of the local time process.
\begin{thm}\la{thm:app2:exist}
Let $\si^2(t) = EG^2(t)$ for $0 \le t \le 1$. If $G$ is a continuous Gaussian process having stationary increment, $G(0) = 0$ and 
\bestar
\int_{0}^{1} \frac{1}{\si(t)} dt < \infty,
\eestar
then the local time $L_G(t, x)$ of the process $G$ exists with probability one.
\end{thm}

Due to $E W_d^2(t) = t^{2d+1}$, where $-1/2<d<1/2$, $W_d$ is a continuous Gaussian process having stationary increment and $W_d(0) = 0$, by Theorem \ref{thm:app2:exist}, the local time for  the class of fractional Brownian motion $W_d$ exists with probability one.

The following theorem gives the distribution of the local time variable, when $B$ is a standard Brownian motion. It comes from \cite{takacs1995}.
\begin{thm} \la{thm:app2:localDist}
Let $\Phi$ be the standard normal distribution function, then
\bestar
P( L_B(1, \al) \le x ) = 2 \Phi(|\al| + x) - 1.
\eestar
\end{thm}
By Theorem \ref{thm:app2:localDist}, it immediately follows that $P(L_B(1, 0) = 0) = 0$ and $P(L_B(1, x) = 0) > 0$ for any fixed $x \ne 0$.


% ------------------------------------------------------------------------

%%% Local Variables: 
%%% mode: latex
%%% TeX-master: "../thesis"
%%% End: 
