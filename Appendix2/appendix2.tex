\chapter{Local Time Process}

This appendix gives the definition and some basic results on the local time of a continuous Gaussian process. The local time up to time $t$ at the point $x$ of a continuous Gaussian process $\{G(t), t\ge 0\}$ can be expressed as the following limit:
\begin{align}
L_G(t,x) &= \lim_{\ep \to 0} \frac{1}{2\ep} \int_{0}^{t} 1_{(x-\ep, x+\ep)}[G(s)] ds \no\\
&= \lim_{\ep \to 0} \frac{1}{2\ep} m\{ s \in [0,t] : G(s) \in (x - \ep, x+\ep) \},
\end{align}
where $m$ is the Lebesgue measure. Such limit can be interpreted as the time spent by the process $G(s)$ at the spatial point $x$ over the time interval $[0, t]$.

The following theorem from Theorem 22.1 of \cite{gemanhorowitz1980} guarantees the existence of the limit.
\begin{thm}
Let $\si^2(t) = EG^2(t)$ for $0 \le t \le 1$. If $G(t)$ has stationary increment, $G(0) = 0$ and 
\bestar
\int_{0}^{1} \frac{1}{\si(t)} dt < \infty,
\eestar
then the local time $L_G(t, x)$ of the process $G(t)$ exists with probability one.
\end{thm}

The following theorem is known as the occupation times formula for a Brownian motion $B(t)$. It comes from Corollary 7.4 of \cite{chungwilliams1990}.
\begin{thm}
For any Borel measurable, locally integrable function $T$ on $\mathbb R$, we have for each fixed $t$,
\bestar
\int_{0}^{t} T[B(s)] ds = \int_{-\infty}^{\infty} T(s) L_B(t, s) ds, \quad \mbox{all } t \in \mathbb R,
\eestar
with probability one.
\end{thm}

The following theorem gives the distribution of the local time variable, when $B(t)$ is a standard Brownian motion. It comes from \cite{takacs1995}.
\begin{thm} \la{thm:app2:localDist}
Let $\Phi(x)$ be the standard normal distribution function, then
\bestar
P( L_B(1, \al) \le x ) = 2 \Phi(|\al| + x) - 1.
\eestar
\end{thm}
By Theorem \ref{thm:app2:localDist}, it immediately follows that $P(L(1, 0) = 0) = 0$ and $P(L(1, x) = 0) > 0$ for any fixed $x \ne 0$.


% ------------------------------------------------------------------------

%%% Local Variables: 
%%% mode: latex
%%% TeX-master: "../thesis"
%%% End: 
